% Copyright 2004 by Till Tantau <tantau@users.sourceforge.net>.
%
% In principle, this file can be redistributed and/or modified under
% the terms of the GNU Public License, version 2.
%
% However, this file is supposed to be a template to be modified
% for your own needs. For this reason, if you use this file as a
% template and not specifically distribute it as part of a another
% package/program, I grant the extra permission to freely copy and
% modify this file as you see fit and even to delete this copyright
% notice. 

\documentclass{beamer}

% There are many different themes available for Beamer. A comprehensive
% list with examples is given here:
% http://deic.uab.es/~iblanes/beamer_gallery/index_by_theme.html
% You can uncomment the themes below if you would like to use a different
% one:
% \usetheme{AnnArbor}
% \usetheme{Antibes}
% \usetheme{Bergen}
% \usetheme{Berkeley}
% \usetheme{Berlin}
% \usetheme{Boadilla}
% \usetheme{boxes}
% \usetheme{CambridgeUS}
% \usetheme{Copenhagen}
\usetheme{Darmstadt}  
% \usetheme{default}
% \usetheme{Frankfurt}
% \usetheme{Goettingen}
% \usetheme{Hannover}
% \usetheme{Ilmenau}
% \usetheme{JuanLesPins}
% \usetheme{Luebeck}
% \usetheme{Madrid}
% \usetheme{Malmoe}
% \usetheme{Marburg}
% \usetheme{Montpellier}
% \usetheme{PaloAlto}
% \usetheme{Pittsburgh}
% \usetheme{Rochester}
% \usetheme{Singapore}
% \usetheme{Szeged}
% \usetheme{Warsaw}

\usepackage{graphicx}
\usepackage{float}
\usepackage{biblatex}
\usepackage{algorithm}
\usepackage[noend]{algpseudocode}
\usepackage{pgfgantt}

\addbibresource{Boole's Rule.bib}
\graphicspath{{./figures/}}

\title{Boole's Rule}

% A subtitle is optional and this may be deleted
\subtitle{Numerical Integration Method}

\author{Harold R. Mansilla\inst{1}}
% - Give the names in the same order as the appear in the paper.
% - Use the \inst{?} command only if the authors have different
%   affiliation.

\institute[University of the Philippines Manila] % (optional, but mostly needed)
{
  \inst{1}%
  Department of Physical Sciences and Mathematics\\
  University of the Philippines Manila
}
% - Use the \inst command only if there are several affiliations.
% - Keep it simple, no one is interested in your street address.

\date{December 5, 2018}
% - Either use conference name or its abbreviation.
% - Not really informative to the audience, more for people (including
%   yourself) who are reading the slides online

% \subject{Theoretical Computer Science}
% This is only inserted into the PDF information catalog. Can be left
% out. 

% If you have a file called "university-logo-filename.xxx", where xxx
% is a graphic format that can be processed by latex or pdflatex,
% resp., then you can add a logo as follows:

\pgfdeclareimage[height=0.5cm]{university-logo}{upm}
\logo{\pgfuseimage{university-logo}}

% Delete this, if you do not want the table of contents to pop up at
% the beginning of each subsection:
\AtBeginSubsection[]
{
  \begin{frame}<beamer>{Outline}
    \tableofcontents[currentsubsection]
  \end{frame}
}

% Let's get started
\begin{document}

\begin{frame}
  \titlepage
\end{frame}

\begin{frame}[allowframebreaks]{Outline}
  \tableofcontents
  % You might wish to add the option [pausesections]
\end{frame}

\section{Background}

\subsection{Newton-Cotes Formulas}

\begin{frame}{Newton-Cotes Formulas}
  \begin{itemize}
    \item The \alert{Newton-Cotes formulas} are a family of \textbf{numerical integration} techniques. \pause 
    \item They are also known as \textit{Quadrature formulas}
  \end{itemize}
\end{frame}

\begin{frame}{Langrange Polynomial Definition}
  The general process of approximating the definite integral is as follows:
    \begin{itemize}
       \item Given a function $f(x)$ to be integrated over some interval $[a,b]$ \pause
       \item Divide the function into $n$ equal parts such that $f_n = f(x_n)$ and $h = \frac{b - a}{n}$
       \begin{itemize}
         \item $f_n$ is the function value at the point $x_n$
         \item $h$ (called the \textit{step size}) is the length of each interval such that $x_{n+1} = x_n + h$
       \end{itemize}
    \end{itemize}
\end{frame}

\begin{frame}{Langrange Polynomial Definition}
  % The general process of approximating the definite integral is as follows:
    \begin{itemize}
        \item Express the function as a \textbf{Langrange Interpolating Polynomial} $P(x)$ \pause
       \begin{equation}
        \begin{aligned}
          \int_{a}^{b} f(x) \mathop{dx} &\approx \int_{a}^{b} P(x) \mathop{dx} \\ \pause
          &\approx \int_{a}^{b} \left(\sum_{i=0}^{n} f(x_i) L_i(x)\right) \mathop{dx} \\ \pause
          &\approx \sum_{i=0}^{n} f(x_i) \int_{a}^{b} L_i(x) \mathop{dx}
        \end{aligned}        
       \end{equation} \pause
       \begin{itemize}
         \item \textit{Note:} $\int_{a}^{b} L_i(x) \mathop{dx}$ can be expressed as $w_i$ or the \textit{weights}
       \end{itemize}
    \end{itemize}
\end{frame}

\begin{frame}{Langrange Polynomial Definition}
  % The general process of approximating the definite integral is as follows:
    \begin{itemize}
      \item Compute the value of the summation.
      \begin{equation} 
        \sum_{i=0}^{n} f(x_i) w_i
      \end{equation}
    \end{itemize}
\end{frame}

\begin{frame}{General Quadrature Formula \cite{general_quadrature}}
  Alternatively, we can express the approximation of the definite integral by \textit{Newton's forward difference interpolating polynomial}. \pause

  Let $f(x_k) = y_k$ be the nodal value at point $x_k$ for $k = 0,1,\dots,x_n$ where $x_0 = a$ and $x_n = x_0 + nh = b$ \pause 

      \begin{equation*}
        \begin{aligned}
          \int_{a}^{b} f(x) \mathop{dx} &\approx \int_a^b P(x) \mathop{dx}
        \end{aligned}
      \end{equation*}
\end{frame}

\begin{frame}{General Quadrature Formula \cite{general_quadrature}}
      \begin{equation*}
        \begin{aligned}
          \int_{a}^{b} f(x) \mathop{dx} &\approx \int_a^b \left[ y_0 + \dfrac{\Delta y_0}{h}\left(x - x_0\right) + \dfrac{\Delta^2 y_0}{2!h^2}\left(x - x_0\right)\left(x - x_1\right) \right. \\ 
          &+ \dfrac{\Delta^3 y_0}{3!h^3}\left(x - x_0\right)\left(x - x_1\right)\left(x - x_2\right) \\
          &+ \left. \dfrac{\Delta^4 y_0}{4!h^4}\left(x - x_0\right)\left(x - x_1\right)\left(x - x_2\right)\left(x - x_3\right) + \dots \right] \mathop{dx}
        \end{aligned}
      \end{equation*}
\end{frame}

\begin{frame}{General Quadrature Formula \cite{general_quadrature}}
  Using the transformation $x = x_0 + hu$ or letting $u = \dfrac{x - x_0}{h}$ and $du = \dfrac{1}{h} \mathop{dx}$
      \begin{equation*}
        \begin{aligned}
          \int_{a}^{b} f(x) \mathop{dx} &\approx h \int_0^n \left[ y_0 + u\Delta y_0 + \dfrac{\Delta^2 y_0}{2!} u\left(u - 1\right) \right. \\
          &+ \dfrac{\Delta^3 y_0}{4!} u\left(u - 1\right)\left(u - 2\right) + \dfrac{\Delta^4 y_0}{4!} u\left(u - 1\right)\left(u - 2\right)\left(u - 3\right) \\
          & + \left. \dots \right] du
        \end{aligned}
      \end{equation*}
\end{frame}

\begin{frame}{General Quadrature Formula \cite{general_quadrature}}
  Integrating the right-hand side, we have:
      \begin{equation*}
        \begin{split}
          \int_{a}^{b} f(x) \mathop{dx} &\approx h\left[ny_0 + \frac{n^2}{2}\Delta y_0 + \frac{\Delta^2 y_0}{2!}\left(\frac{n^3}{3} - \frac{n^2}{2}\right) \right. \\
          &+ \frac{\Delta^3 y_0}{3!}\left(\frac{n^4}{4} - n^3 + n^2\right) \\ 
          &+ \left. \frac{\Delta^4 y_0}{4!}\left(\frac{n^5}{5} - \frac{3n^4}{2} + \frac{11n^3}{3} - 3n^2\right) + \dots \right]
        \end{split}
      \end{equation*}
\end{frame}

\begin{frame}{General Quadrature Formula \cite{general_quadrature}}
  \begin{definition}{General Quadrature Formula}
    \begin{equation}\label{quadrature}
      \begin{split}
        \int_{a}^{b} f(x) \mathop{dx} &\approx h\left[ny_0 + \frac{n^2}{2}\Delta y_0 + \frac{\Delta^2 y_0}{2!}\left(\frac{n^3}{3} - \frac{n^2}{2}\right) \right. \\
        &+ \frac{\Delta^3 y_0}{3!}\left(\frac{n^4}{4} - n^3 + n^2\right) \\ 
        &+ \left. \frac{\Delta^4 y_0}{4!}\left(\frac{n^5}{5} - \frac{3n^4}{2} + \frac{11n^3}{3} - 3n^2\right) + \dots \right]
      \end{split}
    \end{equation}
  \end{definition}
\end{frame}

\begin{frame}{Newton-Cotes Formulas}{Closed Newton-Cotes Formulas}
  \begin{itemize}
    \item A Newton-Cotes formula is \textbf{closed} if it uses the function value at all points
    \begin{itemize}
      \item i.e. The interval may be $[x_0,x_{n-1}]$
    \end{itemize} 
  \end{itemize}
\end{frame}

\begin{frame}{Newton-Cotes Formulas}{Open Newton-Cotes Formulas}
  \begin{itemize}
    \item A Newton-Cotes formula is \textbf{open} if it does not use function values at the endpoints
    \begin{itemize}
      \item i.e. The interval may be $[x_1,x_{n-2}]$
    \end{itemize} 
    \item Will not be discussed further in this report
  \end{itemize}
\end{frame}

\subsection{Closed Newton-Cotes Formulas}

\begin{frame}{Examples of Closed Newton-Cotes Formulas}{Trapezoid Rule}
    2-point closed Newton-Cotes formula $\left(n = 1\right)$
    \begin{equation}
      \int_a^b f(x) \mathop{dx} = \dfrac{h}{2} \left(f_0 + f_1\right)
    \end{equation}
\end{frame}

\begin{frame}{Examples of Closed Newton-Cotes Formulas}{Simpson's 1/3 Rule}
    3-point closed Newton-Cotes formula $\left(n = 2\right)$
    \begin{equation}
      \int_a^b f(x) \mathop{dx} = \dfrac{h}{3} \left(f_0 + 4f_1 + f_2\right)
    \end{equation}
\end{frame}

\begin{frame}{Examples of Closed Newton-Cotes Formulas}{Simpson's 3/8 Rule}
    4-point closed Newton-Cotes formula $\left(n = 3\right)$
    \begin{equation}
      \int_a^b f(x) \mathop{dx} = \dfrac{3h}{8} \left(f_0 + 3f_1 + 3f_2 + f_3\right)
    \end{equation}
\end{frame}

% \begin{frame}{Examples of Closed Newton-Cotes Formulas}{Boole's Rule}
%     5-point closed Newton-Cotes formula $\left(n = 4\right)$
%     \begin{equation}
%       \int_a^b f(x) \mathop{dx} = \dfrac{2h}{45} \left(7f_1 + 32f_2 + 12f_3 + 32f_4 + 7f_5\right)
%     \end{equation}
% \end{frame}

\section{Boole's Rule}

\subsection{History}

\begin{frame}{History \cite{booles_history}}
  \begin{itemize}
    \item Named after \textbf{George Boole}, an English mathematician
    \item Mistakenly called \textit{Bode's Rule} due to propagation of a typographical error
  \end{itemize}
\end{frame}

\subsection{Derivation \cite{booles_quadrature}}

\begin{frame}{Derivation}
  Consider the function $y = f(x)$ over the interval $[x_0,\dots,x_4]$. Boole's rule is an approximation to the integral of $f(x)$ over $[x_0,x_4]$. \\ \pause 
  From Equation \ref{quadrature} with $a = x_0$ and $b = x_4$ \\ \pause 
  Let $n = 4$, such that $f(x)$ can be approximated by a 4th degree polynomial \\
  \begin{equation*}
    \begin{aligned}
      \int_a^b f(x) \mathop{dx} &= \int_{x_0}^{x_4} f(x) \mathop{dx} \\ \pause 
      &= \int_a^{a+4h} f(x) \mathop{dx}
    \end{aligned}
  \end{equation*}
\end{frame}

\begin{frame}{Derivation}
  \begin{equation*}
    \begin{aligned}
      &= h\left[4f(a) + 8\Delta f(a) + \left(\dfrac{64}{3} - 8\right)\dfrac{\Delta^2 f(a)}{2!} + \left(64 - 64 + 16\right)\dfrac{\Delta^3 f(a)}{3!} \right. \\ 
      &+ \left. \left(\dfrac{1024}{5} - 384 + \dfrac{704}{3} - 48\right)\dfrac{\Delta^4 f(a)}{4!}\right] \\
      &= h\left[4f(a) + 8\left\{f(a + h) - f(a)\right\} + \dfrac{20}{3}\left\{f(a + 2h) - 2f(a + h) \right. \right. \\ 
      &+ \left. f(a)\right\} + \dfrac{8}{3} \left\{f(a + 3h) - 3f(a + 2h) + 3f(a + h) - f(a)\right\} \\
      &+ \left. \dfrac{14}{45}\left\{f(a + 4h) - 4f(a + 3h) + 6f(a + 2h) - 4f(a + h) + f(a)\right\} \right] 
    \end{aligned}
  \end{equation*}
\end{frame}

\begin{frame}{Derivation}
  \begin{equation*}
    \begin{aligned}
      &= \dfrac{h}{45}\left[14f(a) + 64f(a + h) + 24(a + 2h) + 64f(a + 3h) + 14f(a + 4h)\right] \\
      &= \dfrac{2h}{45}\left[7f(a) + 32f(a + h) + 12f(a + 2h) + 32f(a + 3h) + 7f(a + 4h)\right] \\ 
      &= \dfrac{2h}{45}\left[7f_0 + 32f_1 + 12f_2 + 32f_3 + 7f_4\right]
    \end{aligned}
  \end{equation*}
\end{frame}

\begin{frame}{Derivation}
  \begin{definition}{Boole's Rule}   
    \begin{equation}\label{booles}
        \int_{x_0}^{x_4} f(x) \mathop{dx} = \dfrac{2h}{45}\left[7f_0 + 32f_1 + 12f_2 + 32f_3 + 7f_4\right]
    \end{equation}
  \end{definition}
\end{frame}

\subsection{Example Comparison \cite{calculator}}

\begin{frame}
  Solve the following using analytic method, Trapezoidal rule, Simpson's 1/3 rule, Simpson's 3/8 rule, and \alert{Boole's Rule}:
  \begin{itemize}
    \item $\int_0^1 \cos{x} \mathop{dx}$
    \item $\int_2^4 \ln{x} \mathop{dx}$
    \item $\int_{-3}^1 6x^2 - 5x + 2 \mathop{dx}$
    \item $\int_{1}^2 x^4 +4x^3 + 9x^2 + 2x + 10 \mathop{dx}$
  \end{itemize}
\end{frame}

\begin{frame}
\begin{table}[]
\centering
\caption{Results}
\label{my-label}
\begin{tabular}{|c|c|l|l|l|}
\hline
\textit{\textbf{Method}} & \multicolumn{4}{c|}{\textit{\textbf{Solution}}}                   \\ \hline
Analytic                 & 0.84147          & 2.15888          & 84          & 55.2          \\ \hline
Trapezoidal              & 0.77015          & 2.07944          & 148         & 62            \\ \hline
Simpson's 1/3            & 0.84177          & 2.15796          & 84          & 55.20833      \\ \hline
Simpson's 3/8            & 0.84160          & 2.15846          & 84          & 55.20370      \\ \hline
\textbf{Boole's}         & \textbf{0.84147} & \textbf{2.15887} & \textbf{84} & \textbf{55.2} \\ \hline
\end{tabular}
\end{table}
\end{frame}

\begin{frame}[t,allowframebreaks]
  \frametitle{References}
  \printbibliography
\end{frame}

\end{document}


