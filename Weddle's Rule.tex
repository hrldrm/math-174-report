% Copyright 2004 by Till Tantau <tantau@users.sourceforge.net>.
%
% In principle, this file can be redistributed and/or modified under
% the terms of the GNU Public License, version 2.
%
% However, this file is supposed to be a template to be modified
% for your own needs. For this reason, if you use this file as a
% template and not specifically distribute it as part of a another
% package/program, I grant the extra permission to freely copy and
% modify this file as you see fit and even to delete this copyright
% notice. 

\documentclass{beamer}

% There are many different themes available for Beamer. A comprehensive
% list with examples is given here:
% http://deic.uab.es/~iblanes/beamer_gallery/index_by_theme.html
% You can uncomment the themes below if you would like to use a different
% one:
% \usetheme{AnnArbor}
% \usetheme{Antibes}
% \usetheme{Bergen}
% \usetheme{Berkeley}
% \usetheme{Berlin}
% \usetheme{Boadilla}
% \usetheme{boxes}
% \usetheme{CambridgeUS}
% \usetheme{Copenhagen}
\usetheme{Darmstadt}  
% \usetheme{default}
% \usetheme{Frankfurt}
% \usetheme{Goettingen}
% \usetheme{Hannover}
% \usetheme{Ilmenau}
% \usetheme{JuanLesPins}
% \usetheme{Luebeck}
% \usetheme{Madrid}
% \usetheme{Malmoe}
% \usetheme{Marburg}
% \usetheme{Montpellier}
% \usetheme{PaloAlto}
% \usetheme{Pittsburgh}
% \usetheme{Rochester}
% \usetheme{Singapore}
% \usetheme{Szeged}
% \usetheme{Warsaw}

\usepackage{graphicx}
\usepackage{float}
\usepackage{biblatex}
\usepackage{algorithm}
\usepackage[noend]{algpseudocode}
\usepackage{pgfgantt}

\addbibresource{Weddle's Rule.bib}
\graphicspath{{./figures/}}

\title{Weddle's Rule}

% A subtitle is optional and this may be deleted
\subtitle{Numerical Integration Method}

\author{Harold R. Mansilla\inst{1}}
% - Give the names in the same order as the appear in the paper.
% - Use the \inst{?} command only if the authors have different
%   affiliation.

\institute[University of the Philippines Manila] % (optional, but mostly needed)
{
  \inst{1}%
  Department of Physical Sciences and Mathematics\\
  University of the Philippines Manila
}
% - Use the \inst command only if there are several affiliations.
% - Keep it simple, no one is interested in your street address.

\date{December 5, 2018}
% - Either use conference name or its abbreviation.
% - Not really informative to the audience, more for people (including
%   yourself) who are reading the slides online

% \subject{Theoretical Computer Science}
% This is only inserted into the PDF information catalog. Can be left
% out. 

% If you have a file called "university-logo-filename.xxx", where xxx
% is a graphic format that can be processed by latex or pdflatex,
% resp., then you can add a logo as follows:

\pgfdeclareimage[height=0.5cm]{university-logo}{upm}
\logo{\pgfuseimage{university-logo}}

% Delete this, if you do not want the table of contents to pop up at
% the beginning of each subsection:
% \AtBeginSubsection[]
% {
%   \begin{frame}<beamer>{Outline}
%     \tableofcontents[currentsubsection]
%   \end{frame}
% }

% Let's get started
\begin{document}

\begin{frame}
  \titlepage
\end{frame}

\begin{frame}[allowframebreaks]{Outline}
  \tableofcontents
  % You might wish to add the option [pausesections]
\end{frame}

\section{Background}

\subsection{Newton-Cotes Formulas}

\begin{frame}{Newton-Cotes Formulas}
  \begin{itemize}
    \item The \alert{Newton-Cotes formulas} are a family of \textbf{numerical integration} techniques. \pause 
    \item They are also known as \textit{Quadrature formulas}
  \end{itemize}
\end{frame}

\begin{frame}{Langrange Polynomial Definition}
  The general process of approximating the definite integral is as follows:
    \begin{itemize}
       \item Given a function $f(x)$ to be integrated over some interval $[a,b]$ \pause
       \item Divide the function into $n$ equal parts such that $f_n = f(x_n)$ and $h = \frac{b - a}{n}$
       \begin{itemize}
         \item $f_n$ is the function value at the point $x_n$
         \item $h$ (called the \textit{step size}) is the length of each interval such that $x_{n+1} = x_n + h$
       \end{itemize}
    \end{itemize}
\end{frame}

\begin{frame}{Langrange Polynomial Definition}
  % The general process of approximating the definite integral is as follows:
    \begin{itemize}
        \item Express the function as a \textbf{Langrange Interpolating Polynomial} $P(x)$ \pause
       \begin{equation} \label{quadrature}
        \begin{aligned}
          \int_{a}^{b} f(x) dx &\approx \int_{a}^{b} P(x) dx \\ \pause
          &\approx \int_{a}^{b} \left(\sum_{i=0}^{n} f(x_i) L_i(x)\right) dx \\ \pause
          &\approx \sum_{i=0}^{n} f(x_i) \int_{a}^{b} L_i(x) dx
        \end{aligned}        
       \end{equation} \pause
       \begin{itemize}
         \item \textit{Note:} $\int_{a}^{b} L_i(x) dx$ can be expressed as $w_i$ or the \textit{weights}
       \end{itemize}
    \end{itemize}
\end{frame}

\begin{frame}{Langrange Polynomial Definition}
  % The general process of approximating the definite integral is as follows:
    \begin{itemize}
      \item Compute the value of the summation.
      \begin{equation}
        \sum_{i=0}^{n} f(x_i) w_i
      \end{equation}
    \end{itemize}
\end{frame}

\begin{frame}{General Quadrature Formula \cite{general_quadrature}}
  Alternatively, we can express the approximation of the definite integral by \textit{Newton's forward difference interpolating polynomial}
    \begin{itemize}
      \item Compute the value of the summation.
      \begin{equation}
        \sum_{i=0}^{n} f(x_i) w_i
      \end{equation}
    \end{itemize}
\end{frame}

\begin{frame}{Newton-Cotes Formulas}{Closed Newton-Cotes Formulas}
  \begin{itemize}
    \item A Newton-Cotes formula is \textbf{closed} if it uses the function value at all points
    \begin{itemize}
      \item i.e. The interval may be $[x_1,x_n]$
    \end{itemize} 
  \end{itemize}
\end{frame}

\begin{frame}{Newton-Cotes Formulas}{Open Newton-Cotes Formulas}
  \begin{itemize}
    \item A Newton-Cotes formula is \textbf{open} if it does not use function values at the endpoints
    \begin{itemize}
      \item i.e. The interval may be $[x_2,x_{n-1}]$
    \end{itemize} 
    \item Will not be discussed further in this report
  \end{itemize}
\end{frame}

\subsection{Closed Newton-Cotes Formulas}

\begin{frame}{Examples of Closed Newton-Cotes Formulas}{Trapezoid Rule}
    2-point closed Newton-Cotes formula $\left(n = 1\right)$
    \begin{equation}
      \int_a^b f(x) dx = \dfrac{h}{2} \left(f_1 + f_2\right)
    \end{equation}
\end{frame}

\begin{frame}{Examples of Closed Newton-Cotes Formulas}{Simpson's Rule}
    3-point closed Newton-Cotes formula $\left(n = 2\right)$
    \begin{equation}
      \int_a^b f(x) dx = \dfrac{h}{3} \left(f_1 + 4f_2 + f_3\right)
    \end{equation}
\end{frame}

\begin{frame}{Examples of Closed Newton-Cotes Formulas}{Simpson's 3/8 Rule}
    4-point closed Newton-Cotes formula $\left(n = 3\right)$
    \begin{equation}
      \int_a^b f(x) dx = \dfrac{3h}{8} \left(f_1 + 3f_2 + 3f_3 + f_4\right)
    \end{equation}
\end{frame}

\begin{frame}{Examples of Closed Newton-Cotes Formulas}{Boole's Rule}
    5-point closed Newton-Cotes formula $\left(n = 4\right)$
    \begin{equation}
      \int_a^b f(x) dx = \dfrac{2h}{45} \left(7f_1 + 32f_2 + 12f_3 + 32f_4 + 7f_5\right)
    \end{equation}
\end{frame}

\section{Weddle's Rule}

\subsection{History}

\begin{frame}{Thomas Weddle (1817-1853) \cite{thomas_weddle}}
  \begin{itemize}
    \item Mathematician who introduced the Weddle surface (unrelated to this report).
    \item Mathematics professor at the Royal Military College at Sandhurst 
  \end{itemize}
\end{frame}

\subsection{Derivation}

\begin{frame}[allowframebreaks]{Derivation}
  Let $n = 6$ \\
  From Equation \ref{quadrature} with $a = x_0$ and $b = x_n$
  \begin{equation*}
    \begin{aligned}
      \int_{x_0}^{x_6} &\approx \int_{x_0}^{x_6} P_{6}(x) dx \\
      &\approx \int_{x_0}^{x_6} \left(\sum_{i=0}^{n} f(x_i) L_i(x)\right) dx \\ 
      &\approx \int_{x_0}^{x_6} f_0 L_0(x) + f_1 L_1(x) + \dots + f_6 L_6(x) dx
    \end{aligned}
  \end{equation*}
  Solving for $L_0,\dots,L_6$
  \begin{equation*}
    \begin{aligned}
      L_0 &= \dfrac{(x - x_0)(x - x_1)(x - x_2)(x - x_3)(x - x_4)(x - x_5)(x - x_6)}{(x_0 - x_1)(x_0 - x_2)(x_0 - x_3)(x_0 - x_4)(x_0 - x_5)(x_0 - x_6)}
      % &= h\dfrac{}{(-h)(-2h)(-3h)(-4h)(-5h)(-6h)}
    \end{aligned}
  \end{equation*}
\end{frame}

\section{References}

\begin{frame}[t,allowframebreaks]
  \frametitle{References}
  \printbibliography
\end{frame}

\end{document}


